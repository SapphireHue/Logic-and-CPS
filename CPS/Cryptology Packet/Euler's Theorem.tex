\documentclass{article}
\usepackage{graphicx} % Required for inserting images
\usepackage[a4paper, margin=1in]{geometry}
\usepackage{amsmath}
\usepackage{amssymb}

\begin{document}
\begin{flushleft}
    \LARGE\textbf{Euler's Theorem}\normalsize\\ \vspace{11pt}
    First, we’ll define a new function called the Euler phi-function or the Euler totient-function, $\phi(n)$. If n is a positive integer, then $\phi(n)$ represents the number of positive integers not exceeding n that are relatively prime to $n$. You should verify that $\phi(8) = 4$.\\
    
    \textbf{Theorem 4.1 Euler's Theorem:} If $m$ is a positive integer and $a$ is a positive integer with $\gcd(a, m) = 1$, then $a^{\phi(n)} \equiv 1 \mod{m}$
\end{flushleft}

\begin{enumerate}
    \item Find $\phi(n)$ for the following integers.
        \begin{enumerate}
            \item 7\\
            The following numbers are relatively prime to 7: 1, 2, 3, 4, 5, 6. Therefore, $\phi(7) = 6$.
            \item 10\\
            The following numbers are relatively prime to 10: 1, 3, 7, 9. Therefore, $\phi(10)=4$.
            \item 11\\
            The following numbers are relatively prime to 11: 1, 2, 3, 4, 5, 6, 7, 8, 9, 10. Therefore, $\phi(11) = 10$.
            \item 16\\
            The following numbers are relatively prime to 16: 1, 3, 5, 7, 9, 11, 13, 15. Therefore, $\phi(16) = 8$.
        \end{enumerate}
    \item Find the last digit in the decimal expansion of $3^{1000}$.\\
        We're trying to find $3^{1000} \mod 10$. $\phi(10) = 4$. $3^{1000} = (3^{4})^{250} \equiv 1^{250} = 1 \mod{10}$. The last digit is 1.
    \item Find the last digit in the decimal expansion of $7^{999,999}$.\\
    $7^{999,999} = 7^{1,000,000} \times 7^{-1} = (7^{4})^{250,000} \times 7^{-1}$. We're still in mod 10, so $(7^{4})^{250,000} \times 7^{-1} \equiv 1^{250,000} \times 7^{-1} = 1 \times 7^{-1} = 7^{-1}$. By iterating from numbers 1 to 7, I found that the multiplicative inverse of 7 mod 10 is 3, so the last digit is 3.
    \item Find the number in $\mathbb{Z}_{35}$ congruent to $3^{100,000}$.\\
    The following numbers are relatively prime to 35: 1, 2, 3, 4, 6, 8, 9, 11, 12, 13, 16, 17, 18, 19, 22, 23, 24, 26, 27, 29, 31, 32, 33, 34. Thus, $\phi(35) = 24$. $3^{100,000} = (3^{24})^{4000} \times 3^{4000} \equiv 1^{4000} \times 3^{4000} = 3^{4000} = (3^{24})^{160} \times 3^{160} \equiv 1^{160} \times 3^{160} = (3^{24})^{6} \times 3^{16} \equiv 1^{6} \times 3^{16} = (3^{5})^{3} \times 3 = 243^{3} \times 3 \equiv (-2)^{3} \times 3 = -24 \equiv 11 \mod 35$. Thus, $3^{100,000} \equiv 11 \mod{35}$.
    \item Use Euler’s Theorem to find the multiplicative inverse of 2 modulo 9.\\
    9 is relatively prime to: 1, 2, 4, 5, 7, 8. $\phi(9) = 6$. Therefore, $2^{6} = 2 \times 2^{5} \equiv 1 \mod{9}$. Therefore, the multiplicative inverse of 2 mod 9 is $2^{5}$. $2^{5} = 32 \equiv 5$, so 5 is the multiplicative inverse of 2 modulo 9.
    \item Solve each of the following linear congruences using Euler’s Theorem.
    \begin{enumerate}
        \item $5x \equiv 3 \mod{14}$\\
        $x = 5^{-1} \times 3 \mod{14}$. 14 is relatively prime to: 1, 3, 5, 9, 11, 13. Therefore, $\phi(14) = 6$, so $5^{6} \equiv 1 \mod{14}$, so $5^{-1} \equiv 5^{5} \mod{14}$. $5^{5} = 25 \times 25 \times 5 \equiv -3 \times -3 \times 5 = 45 \equiv 3$. So, $5^{-1} \equiv 3 \mod{14}$, so $x = 3 \times 3 = 9$.
        \item $4x \equiv 7 \mod{15}$\\
        $x = 4^{-1} \times 7 \mod{15}$. 15 is relatively prime to: 1, 2, 4, 7, 8, 11, 13, 14. Therefore, $\phi(15) = 8$, so the multiplicative inverse of 4 modulo 15 is $4^{7}$. $4^{7} = (4^{2})^{3} \times 4 = 16^{3} \times 4 \equiv 1^{3} \times 4 = 4$. Therefore, $x = 4 \times 7 = 28 \equiv 13 \mod{15}$. $x = 13$.
        \item $3x \equiv 5 \mod{16}$\\
        $x = 3^{-1} \times 5 \mod{16}$. 16 is relatively prime to: 1, 3, 5, 7, 9, 11, 13, 15. Therefore, $\phi(16) = 8$, so the multiplicative inverse of 3 modulo 16 is $3^{7}$. $3^{7} = 3^{3} \times 3^{3} \times 3 = 27 \times 27 \times 3 \equiv 11 \times 11 \times 3 = 11 \times 33 \equiv 11 \times 1 = 11$. $x = 11 \times 5 = 55 \equiv 7 \mod{16}$. $x = 7$.
    \end{enumerate}
    \item If $p$ and $q$ are distinct primes, what is $\phi(pq)$? It is safe to assume that $\phi$ is a multiplicative function (i.e., $\phi(pq) = \phi(p) \cdot \phi(q)$) if $p$ and $q$ are distinct primes.\\
    Because $p$ is prime, only 1 and $p$ divide it evenly. Thus, for every number $n$ such that $0 < n < p$, $\gcd(n, p) = 1$. $n$ can be any integer in the range $[1, p-1]$, so there are $p - 1$ numbers less than and relatively prime to $p$, so $\phi(p) = p - 1$. By the same reasoning, $\phi(q) = q - 1$. Since $\phi$ is multiplicative, $\phi(pq) = \phi(p) \cdot \phi(q) = (p - 1)(q - 1)$.
\end{enumerate}
\end{document}
