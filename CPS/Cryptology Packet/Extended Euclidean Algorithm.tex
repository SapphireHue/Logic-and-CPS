\documentclass{article}
\usepackage[a4paper, margin=1in]{geometry}
\usepackage{graphicx}
\usepackage{amsmath}
\usepackage{amssymb}
\usepackage{xcolor}
\usepackage{minted}
\begin{document}
\begin{flushleft}
    \LARGE\textbf{The Extended Euclidean Algorithm}\normalsize\\ \vspace{11pt} 
    \textbf{Background}\\
    \textbf{Bézout’s Identity}: There exist integers $x, y$ such that $ax + by = \gcd(a, b)$.\\
    Let $a = E$ and $b = (p-1)(q-1)$. In the RSA cryptosystem, $E$ must be relatively prime to $(p-1)(q-1)$. Thus, there exist integers $x, y$ such that $Ex + (p-1)(q-1)y = 1$. In $\mod{(p-1)(q-1)}$: $Ex + (p-1)(q-1)y \equiv 1 \mod{(p-1)(q-1)}$, so $Ex \equiv 1 \mod{(p-1)(q-1)}$, so $x = E^{-1} = D$.\\ \vspace{8pt}   
    
    \textbf{Theorem 6.2}: If $a$ and $b$ are integers such that $a > b > 0$, and $q$ and $r$ are integers such that $a = qb + r$ where $0 \leq r < b$ (Euclidean Division), then $\gcd(a, b) = \gcd(b, r)$. i.e. $\gcd(a, b) = \gcd(a-b, b)$.\\
    The Euclidean Algorithm applies Euclidean Division repeatedly to find the gcd by reducing the problem to smaller pieces.\\\vspace{8pt}   

    The Extended Euclidean Algorithm finds the values of $x$ and $y$ specified in Bézout’s Identity by reversing the steps of the Euclidean Algorithm.\\
    Example: Finding $gcd(97, 20)$, and then $x, y$ such that $97x + 20y = gcd(97, 20)$.
    $\begin{array}{|rclclr|rclclcrclr|}
        \hline
         \multicolumn{6}{|c|}{\textbf{The Euclidean Algorithm}}& \multicolumn{10}{|c|}{\textbf{The Extended Euclidean Algorithm}}\\ \hline
          97&=&4\cdot20&+&17&\text{(Step 1)}&1&=&3     &-&1(2)           &=&1(3)  &-&1(2)&\text{(Rearrange step 4)}\\  
          20&=&1\cdot17&+&3 &\text{(Step 2)}& &=&1(3)  &-&1(17-5\cdot3)&=&-1(17)&+&6(3)&\text{(Sub from step 3)}\\
          17&=&5\cdot3 &+&2 &\text{(Step 3)}& &=&-1(17)&+&6(20-1\cdot17)&=&6(20)&-&7(17)&\text{(Sub from step 2)}\\
          3 &=&1\cdot2 &+&1 &\text{(Step 4)}& &=&6(20)&-&7(97-4\cdot20)&=&-7(97)&+&34(20)&\text{(Sub from step 1)}\\
          2 &=&2\cdot1 &+&0 &\text{(Step 5)}&&&&&&&&&&\\
         \hline
    \end{array}$
    Thus, $\gcd(97, 20) = 1$ and $97(-7) + 20(34) = 1$.
\end{flushleft}
\hrule \vspace{8pt}   
\textbf{Problem Set}
    \begin{enumerate}
        \item Use the Euclidean Algorithm to find the greatest common divisor for each of the following pairs of integers without using a computer.
        \begin{enumerate}
            \item 51, 89\\
            $89 = 1\cdot51 + 38 \to 51 = 1\cdot38 + 13 \to 38 = 2\cdot13 + 12 \to 13 = 1\cdot12 + 1 \to 12 = 1\cdot12 + 0$. Thus, $\gcd(51, 89) = \gcd(89, 51) = \gcd(51, 38) = \gcd(38, 13) = \gcd(13, 12) = \gcd(12, 1) = 1$.
            \item 102, 202\\
            $202 = 1\cdot102 + 100 \to 102 = 1\cdot100 + 2 \to 100 = 50\cdot2 + 0$. Thus, $\gcd(102, 202) = 2$.
            \item 666, 1414\\
            $1414 = 2\cdot666 + 82 \to 666 = 8\cdot82 + 10 \to 82 = 8\cdot10 + 2 \to 10 = 5\cdot2 + 0$. Thus, $\gcd(666, 1414) = 2$.
        \end{enumerate}
        \item Use the Extended Euclidean Algorithm to express the greatest common divisor of each of the following pairs of integers as a linear combination of these integers.
        \begin{enumerate}
            \item 51, 89\\
            $1 = 1(13) - 1(12) = 1(13) - 1(38 - 2\cdot13) = -1(38) + 3(13) = -1(38) + 3(51-1\cdot38) = 3(51) - 4(38) = 3(51) - 4(89-1\cdot51) = -4(89) + 7(51)$.
            \item 102, 202\\
            $2 = 1(102) - 1(100) = 1(102) - 1(202 - 1\cdot102) = -1(202) + 2(102)$.
            \item 666, 1414\\
            $2 = 1(82) - 8(10) = 1(82) - 8(666 - 8\cdot82) = -8(666) + 65(82) = -8(666) + 65(1414-2\cdot666) = 65(1414) - 138(666)$.
        \end{enumerate}
        \item Find $50^{-1} \mod{127}$ without using a computer.\\
        By Bézout’s Identity, if $\gcd(50, 127) = 1$, then $50x + 127y = 1$, so $x = 50^{-1} \mod{127}$. \\
        First, we use the Euclidean algorithm to check $gcd(50, 127)$. $127 = 2\cdot50 + 27 \to 50 = 1\cdot27 + 23 \to 27 = 1\cdot23 + 4 \to 23 = 5\cdot4 + 3 \to 4 = 1\cdot3 + 1 \to 3 = 3\cdot1 + 0$. Thus, $gcd(50, 127) = 1$.\\
        Then, we use the Extended Euclidean Algorithm to solve for $x$ and $y$. $1 = 1(4) - 1(3) = 1(4) - 1(23 - 5\cdot4) = -1(23) + 6(4) = -1(23) + 6(27 - 1\cdot23) = 6(27) - 7(23) = 6(27) - 7(50-1\cdot27) = -7(50) + 13(27) = -7(50) + 13(127 - 2\cdot50) = 13(127) - 33(50)$. Thus, $50^{-1}\mod{127} = -33$.
        \item In the RSA cryptosystem, find $D$ given $p = 13$, $q = 17$, and $E = 5$.\\
        $(p-1)(q-1) = 12\cdot16 = 192$. Now, we apply the Extended Euclidean Algorithm on $192, 5$ to find $D = 5{-1} \mod{192}$.\\
        The Euclidean Algorithm: $192 = 38\cdot5 + 2 \to 5 = 2\cdot2 + 1 \to 2 = 2\cdot1 + 0$.\\
        The extension: $1 = 1(5) - 2(2) = 1(5) - 2(192 - 38\cdot5) = -2(192) + 77(5)$. Thus, $D = 77$.
        \item Write a Python function that uses the Euclidean Algorithm to find and return the greatest common divisor of two positive integers. Use your function to find the greatest common divisor for each of the following pairs of integers.
        \begin{minted}{python}
def euclideanGCD(a, b):
    equations = []
    if(a < b):
        a, b = b, a
    r = -1
    while(r!=0):
        q = a // b
        r = a - q * b
        equations.append({'a':a, 'q':q, 'b':b, 'r':r})
        a, b = b, r
    print("gcd(" + str(equations[0]['a']) + ", " + 
        str(equations[0]['b']) + ") = " + str(equations[-1]['b']))
    return equations
        \end{minted}
        \begin{enumerate}
            \item $\gcd(9876543210, 123456789) = 9$
            \item $\gcd(11111111111, 1000000001) = 1$
            \item $\gcd(73433510078091009, 45666020043321) = 3$
        \end{enumerate}
        \item Write a Python function that uses the Extended Euclidean Algorithm to write the greatest common divisor of each of the following pairs of positive integers as a linear combination of those integers.
        \begin{minted}{python}
def extendedAlgorithm(a, b):
    equations = euclideanGCD(a, b)
    g = equations[-1]['b']
    if len(equations) < 2:
        print(str(g) + " = " + str(equations[0]['a']) + "*0 + "
            + str(equations[0]['b']) + "*1") 
        return {'gcd':g, 'a':equations[0]['a'], 'x':0, 'b':equations[0]['b'], 'y':1}
    equations=equations[-2::-1]
    a = 1
    x = equations[0]['a']
    b = -equations[0]['q']
    y = equations[0]['b']
    for i in range(1, len(equations)):
        newA = b
        newX = equations[i]['a']
        newB = a + b * -equations[i]['q']
        newY = x
        a, x, b, y = newA, newX, newB, newY
    print(str(g) + " = " + str(a) + "*" + str(x) + " + " + str(b) + "*" + str(y)) 
    return {'gcd':g, 'a':a, 'x':x, 'b':b, 'y':y}
        \end{minted}
        \begin{enumerate}
            \item $9 = -1371742\cdot9876543210 + 109739361\cdot123456789$
            \item $1 = 99009901\cdot11111111111 + -1100110010\cdot1000000001$
            \item $3 = -6495686882417\cdot73433510078091009 + 10445427205865236\cdot45666020043321$
        \end{enumerate}
    \item In the previous section, we used the Extended Euclidean Algorithm to find that $34 \equiv 20^{-1} \mod{97}$.
    \begin{enumerate}
        \item Find $20^{-1} \mod{97}$ by making clever use of Euler’s Theorem.\\
        By Euler's theorem, $20^{\phi(97)} \equiv 1 \mod{97}$, so $20^{\phi(97)-1} \equiv 20^{-1} \mod{97}$. $97$ is prime, so $\phi(97) = 96$. Thus, $20^{-1} \equiv 20^(96-1) = 20^{95} \mod{97}$. $20^{95} = (20^{2})^{47} \times 20 = 400^{47} \times 20 \equiv 12^{47} \times 20 = (12^{3})^{15} \times 144 \times 20 = 1728^{15} \times 2880 \equiv (-18)^{15} \times -30 = ((-18)^{2})^{7} \times -18 \times -30 = 324^{7} \times 540 \equiv 33^{7} \times 55 = (33^{2})^{3} \times 33 \times 55 = 1089^{3} \times 1815 \equiv 22^{3} \times 69 = 10648 \times 69 \equiv 75 \times 69 = 5175 \equiv 34$.
        \item Explain why using Euler’s Theorem is be a much more computationally difficult strategy than using the Extended Euclidean Algorithm.\\
        Euler's theorem first requires finding the Euler-Totient function of the modulus (let's call it $m$), which in turn requires calculating $m-1$ GCDs. It then involves raising the number you're trying to find the inverse of ($E$) to the power of $m$, which involves either many multiplications and modulo operations, or multiplying very large numbers together.\\
        The Extended Euclidean Algorithm, on the other hand, only needs to calculate a GCD once (and is a calculation that would be performed in calculating $\phi(m)$ for Euler's theorem anyway, since $E < m$). The program uses a little bit more memory to drastically cut down on the number of operations it has to perform.
    \end{enumerate}
    \end{enumerate}
\end{document}