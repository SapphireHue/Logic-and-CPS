\documentclass{article}
\usepackage[a4paper, margin=1in]{geometry}
\usepackage{amssymb}
\usepackage{amsmath}
\begin{document}
\begin{flushleft}
      \LARGE\textbf{Modular Numbers}\normalsize\\ \vspace{11pt}

      Two integers, $m$ and $n$, are said to be congruent modulo $d$ if they have the same remainder when divided by $d$. This is written $m \equiv n \mod{d}$.\\ \vspace{11pt}

      Consider integers mod 4. Regardless of which integer you pick, that integer must be congruent to 0, 1, 2, or 3 modulo 4 because those are the only possibilities for remainders after dividing by 4. So, for example, we say that all of the numbers in ${..., -3, 1, 5, 9,...}$ belong to the same congruence class mod 4. It would be nice to pick just one representative of that congruence class, and nicest of all to make that representative the smallest non-negative member of the equivalence class. So, we would say that the \textit{canonical complete residue system} modulo 4 is {0, 1, 2, 3} or $\mathbb{Z}_{4}$.\\ \vspace{11pt}

      Henceforth, $\mathbb{Z}_{n}$ will refer to the set of integers from 0 to $n-1$, which represents all of the congruence classes modulo $n$.
\end{flushleft}

\begin{enumerate}
      \item Consider the integers modulo 6.
            \begin{enumerate}
                  \item Construct a table for addition modulo 6.\\
                        \begin{tabular}{c|c|c|c|c|c|c|c|c}
                              \textbf{+} & \textbf{0} & \textbf{1} & \textbf{2} & \textbf{3} & \textbf{4} & \textbf{5} \\ \hline
                              \textbf{0} & 0          & 1          & 2          & 3          & 4          & 5          \\ \hline
                              \textbf{1} & 1          & 2          & 3          & 4          & 5          & 0          \\ \hline
                              \textbf{2} & 2          & 3          & 4          & 5          & 0          & 1          \\ \hline
                              \textbf{3} & 3          & 4          & 5          & 0          & 1          & 2          \\ \hline
                              \textbf{4} & 4          & 5          & 0          & 1          & 2          & 3          \\ \hline
                              \textbf{5} & 5          & 0          & 1          & 2          & 3          & 4          \\ \hline
                        \end{tabular}
                  \item Construct a table for subtraction modulo 6.\\
                        \begin{tabular}{c|c|c|c|c|c|c|c|c}
                              $-$        & \textbf{0} & \textbf{1} & \textbf{2} & \textbf{3} & \textbf{4} & \textbf{5} \\ \hline
                              \textbf{0} & 0          & 1          & 2          & 3          & 4          & 5          \\ \hline
                              \textbf{1} & 5          & 0          & 1          & 2          & 3          & 4          \\ \hline
                              \textbf{2} & 4          & 5          & 0          & 1          & 2          & 3          \\ \hline
                              \textbf{3} & 3          & 4          & 5          & 0          & 1          & 2          \\ \hline
                              \textbf{4} & 2          & 3          & 4          & 5          & 0          & 1          \\ \hline
                              \textbf{5} & 1          & 2          & 3          & 4          & 5          & 0          \\ \hline
                        \end{tabular}
                  \item Construct a table for multiplication modulo 6.\\
                        \begin{tabular}{c|c|c|c|c|c|c|c|c}
                              $\times$   & \textbf{0} & \textbf{1} & \textbf{2} & \textbf{3} & \textbf{4} & \textbf{5} \\ \hline
                              \textbf{0} & 0          & 0          & 0          & 0          & 0          & 0          \\ \hline
                              \textbf{1} & 0          & 1          & 2          & 3          & 4          & 5          \\ \hline
                              \textbf{2} & 0          & 2          & 4          & 0          & 2          & 4          \\ \hline
                              \textbf{3} & 0          & 3          & 0          & 3          & 0          & 3          \\ \hline
                              \textbf{4} & 0          & 4          & 2          & 0          & 4          & 2          \\ \hline
                              \textbf{5} & 0          & 5          & 4          & 3          & 2          & 1          \\ \hline
                        \end{tabular}
            \end{enumerate}
      \item Which decimal digits occur as the final digit of a fourth power of an integer?\\
            $0^{4} = 0, 1^{4} = 1, 2^{4} = 16, 3^{4} = 81, 4^{4} = 256, 5^{4} = 625, 6^{4} = 1296, 7^{4} = 2401, 8^{4} = 4096, 9^{4} = 6561$\\
            $0, 1, 5, \text{and}\ 6$ can occur as the final digit of the fourth power of an integer.
      \item Compute the number $k$ in $\mathbb{Z}_{12}$ such that $37^{453} \equiv k \mod{12}$. Explain how you did it.\\
            $(37^{453} = 37 \times 37 \times 37 \times ...)$ and $((A \times B)\ \text{mod}\ C = (A\ \text{mod}\ C \times B\ \text{mod}\ C)\ \text{mod}\ C)$. Therefore, $37^{453}\ \text{mod}\ 12 = (37\ \text{mod}\ 12)^{453}\ \text{mod}\ 12 = 1^{453}\ \text{mod}\ 12 = 1$. So, $37^{453} \equiv 1 \mod{12}$.
      \item Compute the number $k$ in $\mathbb{Z}_{7}$ such that $2^{50} \equiv k \mod{7}$ without using a computer.\\
            $2^{50} = 2^{48} \times 2^{2} = (2^{3})^{16} \times 2^{2}$. Therefore, $2^{50}\ \text{mod}\ 7 = ((8^{16}\ \text{mod}\ 7) \times 4)\ \text{mod}\ 7 = 1^{16} \times 4\ \text{mod}\ 7 = 4\ \text{mod}\ 7$. So, $2^{50} \equiv 4 \mod{7}$.
      \item Compute the number $k$ in $\mathbb{Z}_{7}$ such that $39^{453} \equiv k \mod{12}$ without using a computer.\\
            % $39^{453} \equiv 3^{453} \equiv (3^{3})^{151} \equiv 27^{151} \equiv 3^{151} \equiv 27^{50} \times 3 \equiv 3^{50} \times 3 \equiv 27^{16} \times 3^{3} \equiv 3^{16} \times 3^{3} \equiv 27^{5} \times 3^{4} \equiv 3^{5} \times 3^{3} \equiv 3^{9} \equiv 27^{3} \equiv 3^{3} \equiv 27 \equiv 3 \mod{12}$. So, $39^{453} \equiv 3 \mod{12}$.\\
            $39^{453} \equiv 3^{453} \equiv (3^{3})^{151} \equiv 27^{151} \equiv 3^{151} \equiv (3^{3})^{50} \times 3 \equiv 27^{50} \times 3 \equiv 3^{50} \times 3 \equiv 3^{51} \equiv 27^{27} \equiv 3^{27} \equiv 27^{9} \equiv 3^{9} \equiv 27^{3} \equiv 3^{3} \equiv 27 \equiv 3$. So, $39^{453} \equiv 3 \mod{12}$.
      \item Find the numbers in $\mathbb{Z}_{47}$ that are congruent to each of the following without using a computer:
            \begin{enumerate}
                  \item $2^{32}$\\
                        $2^{32} = (2^{8})^{4} = 256^{4} \equiv 21^{4} = 441^{2} \equiv 18^{2} = 324 \equiv 42 \mod{47}$. So, $2^{32} \equiv 42 \mod{47}$.
                  \item $2^{47}$\\
                        $2^{47} = 2^{32} \times 2^{15} \equiv 42 \times 2^{15} = 21 \times 2^{16} = 21 \times (2^{8})^{2} = 21 \times 256^{2} \equiv 21 \times 21^{2} = 21 \times 441 \equiv 21 \times 18 = 378 \equiv 2 \mod{47}$. So, $2^{47} \equiv 2 \mod{47}$.
                  \item $2^{200}$\\
                        $2^{200} = (2^{47})^{4} \times 2^{12} \equiv 2^{4} \times 2^{12} = 2^{16} = 256^{2} \equiv 21^{2} = 441 \equiv 18 \mod{47}$. So, $2^{200} \equiv 18 \mod {47}$
            \end{enumerate}
      \item Find the canonical residue congruent to each of the following without using a computer.
            \begin{enumerate}
                  \item $3^{10} \mod{11}$\\
                        $3^{10} = (3^{3})^{3} \times 3 = 27^{3} \times 3 \equiv 5^{3} \times 3 = 25 \times 5 \times 3 = 25 \times 15 \equiv 3 \times 4 = 12 \equiv 1 \mod{11}$.
                  \item $2^{12} \mod{13}$\\
                        $2^{12} = (2^{4})^{3} = 16^{3} \equiv 3^{3} = 27 \equiv 1 \mod{13}$.
                  \item $5^{16} \mod{17}$\\
                        $5^{16} = (5^{2})^{8} = 25^{8} \equiv 8^{8} = (2^{3})^{8} = 2^{24} = (2^{4})^{6} = 16^{6} \equiv (-1)^{6} = 1 \mod{17}$.
                  \item $3^{22} \mod{23}$\\
                  $3^{22} = (3^{3})^{7} \times 3 = 27^{7} \times 3 \equiv 4^{7} \times 3 = (4^{3})^{2} \times 4 \times 3 = 64^{2} \times 12 \equiv (-5)^{2} \times 12 = 25 \times 12 \equiv 2 \times 12 = 24 \equiv 1 \mod{23}$.
                  \item Make a conjecture based on the congruences in this problem.\\
                  For $n, x \in \mathbb{Z}$, $x^{n+1} \equiv 1 \mod{n}$.
            \end{enumerate}
\end{enumerate}
\end{document}