\documentclass{article}
\usepackage[a4paper, margin=1in]{geometry}
\usepackage{amsmath}
\begin{document}
\begin{flushleft}
      \LARGE\textbf{Greatest Common Divisor}\normalsize\\ \vspace{11pt}

      The \textit{greatest common divisor} (GCD) of two integers, $a$ and $b$, which are not both 0, is the largest integer that divides both $a$ and $b$. We write this gcd($a, b$) and define gcd($0, 0$) to be 0 because otherwise it would be annoying.\\ \vspace{11pt}

      We also have a name for two integers that share no factors. If gcd($a, b$) = 1, we say that $a$ and $b$ are relatively prime.
\end{flushleft}

\begin{enumerate}
      \item Find the greatest common divisor for each of the following pairs of integers.
            \begin{enumerate}
                  \item 15, 35\\
                        5
                  \item 0, 111\\
                        111 (0 is divisible by every integer)
                  \item -12, 18\\
                        6
                  \item 99, 100\\
                        1 (the prime factors of 99 are 3, 3, and 11; the prime factors of 100 are 2, 2, 5, and 5)
                  \item 11, 121\\
                        11
                  \item 100, 102\\
                        2
            \end{enumerate}
      \item Let $a$ be a positive integer. What is gcd($a, 2a$)?\\
            $a$
      \item Let $a$ be a positive integer. What is gcd($a, a^{2}$)?\\
            $a$
      \item Let $a$ be a positive integer. What is gcd($a, a + 1$)?\\
            1. % $\gcd(a, b) \leq \min(a, b)$, so $\gcd(a, a+1) \leq a$. Every positive integer can be divided by 1, so $1 \leq \gcd(a, a+1) \leq a$. Every divisor 
            Every integer can be divided by 1, so $1 \leq \gcd(a, a+1)$. Let $a$ have a divisor $x$ such that $x > 1$. $a\ \text{mod}\ x = 0$, so $(a+1)\ \text{mod}\ x = 1$. Thus, there is no integer greater than 1 that divides both $a$ and $a+1$.
      \item Let $a$ be a positive integer. What is gcd($a, a + 2$)?\\
            If 2 divides $a$, $\gcd(a, a+2) = 2$.\\
            Otherwise, $\gcd(a, a+2) = 1$.
      \item Find the greatest common divisor for each of the following sets of integers.
            \begin{enumerate}
                  \item 8, 10, 12\\
                        2
                  \item 6, 15, 21\\
                        3
                  \item -7, 28, -35\\
                        7
            \end{enumerate}
      \item Find a set of three integers that are mutually relatively prime, but any two of which are not  relatively prime.\\
            6, 15, 10\\
            To achieve this task, I figured I needed 3 numbers, each with 2 distinct prime factors. Each pair of numbers shares a prime factor, so I then reverse-engineered this. The 3 smallest prime numbers are 2, 3, and 5, so I multiplied pairs of these together.
      \item Find four integers that are mutually relatively prime such that any three of these integers are not mutually relatively prime.\\
      105, 70, 42, 30\\ 
      This was found with a similar process. From 4 items, I could construct 4 groups of 3 (each group consists of 1 from the initial group removed), so I wanted 4 unique prime factors; I chose 2, 3, 5, and 7. I constructed 4 groups of 3: \{3, 5, 7\}, \{2, 5, 7\}, \{2, 3, 7\}, and \{2, 3, 5\}. I then took the product of each group.
\end{enumerate}
\end{document}