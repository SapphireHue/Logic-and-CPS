\documentclass{article}
\usepackage[a4paper, margin=1in]{geometry}
\usepackage{graphicx} % Required for inserting images
\usepackage{amsmath}
\usepackage{amssymb}
\usepackage{hyperref}

\begin{document}
\begin{flushleft}
    \LARGE\textbf{The RSA Cryptosystem}\normalsize\\ \vspace{11pt}
    \textbf{The Theoretical Basis for RSA Encryption}\\
    The RSA algorithm involves 5 numbers: $p, q, E, D,$ and $M$. As a brief introduction, they are:
    \begin{itemize}
        \item two different prime numbers, $p$ and $q$
        \item two numbers in $\mathbb{Z}_{(p-1)(q-1)}$:
              \begin{itemize}
                  \item the "encoding number", $E$, which is relatively prime to $(p-1)(q-1)$
                  \item the "decoding number", $D$, which is the multiplicative inverse of $E$ in $\mathbb{Z}_{(p-1)(q-1)}$
              \end{itemize}
        \item a number in $\mathbb{Z}_pq$ known as the "message number", $M$
    \end{itemize}
\end{flushleft}

\begin{enumerate}
    \item Do the activity \href{https://docs.google.com/spreadsheets/d/1vwZHtPWf8RX4x7-8a8qORGgtC9CZb5UV/edit?usp=sharing&ouid=103144376998261263253&rtpof=true&sd=true}{Modular Inverses}\\
          $E$ must be relatively prime to $(p-1)(q-1)$ in order to have a multiplicative inverse in the system.
    \item Let $p=3$ and $q=5$. Then $(p-1)(q-1) = 8$ and $pq = 15$. Suppose that we choose $E$ to be 3. (We could choose any number that didn’t have a factor of 2 since 2 is the only factor in 8). Find $D$.\\
          $D$ is the multiplicative inverse of $E$ in $\mathbb{Z}_{8}$. Thus, $D = 3$.
    \item Sticking with the same $p, q,$ and $E$ (and therefore the same $D$), complete the table below using the rules of $\mathbb{Z}_{15}$. What do you notice about the entries of the last row?\\
          \begin{tabular}{|c||c|c|c|c|c|c|c|c|c|c|c|c|c|c|}
              \hline
              $M \mod{pq}$      & 1 & 2 & 3  & 4 & 5 & 6 & 7  & 8 & 9 & 10 & 11 & 12 & 13 & 14 \\
              \hline
              \hline
              $M^{E} \mod{pq}$  & 1 & 8 & 12 & 4 & 5 & 6 & 13 & 2 & 9 & 10 & 11 & 3  & 7  & 14 \\
              \hline
              $M^{ED} \mod{pq}$ & 1 & 2 & 3  & 4 & 5 & 6 & 7  & 8 & 9 & 10 & 11 & 12 & 13 & 14 \\
              \hline
          \end{tabular}\\
          The last row is the same as the first row.
    \item \href{https://docs.google.com/spreadsheets/d/1TyXS6F61i7I05NOypTlqcOyrf-DUD3SJwcPTzKUEWaU/edit?usp=sharing}{Google Sheets RSA Cryptosystem Crux Theorem Activity}\\
          From the activity, it seems that raising a number to a power, then raising that to the multiplicative inverse of the previous power, produces the original number.
\end{enumerate}
\begin{flushleft}
    \textbf{Theorem 5.1 (RSA Cryptosystem Crux):} Suppose that $p$ and $q$ are two distinct prime numbers. Let $E$ be relatively prime to $(p-1)(q-1)$. Then if $D$ is the multiplicative inverse of $E$ in $\mathbb{Z}_{(p-1)(q-1)}$ and if $M$ is any number in $\mathbb{Z}_{pq}$, it will always be that $M^{ED} \equiv M \mod{pq}$.\\

    \textbf{Theorem 5.2:} If $p$ and $q$ are distinct prime numbers and $M$ is a positive integer with $\gcd(M, pq) = 1$, then $M^{(p-1)(q-1)} \equiv 1 \mod{pq}$.\\

    \textbf{Theorem 5.3:} Let $p$ and $q$ be distinct prime numbers, $k$ be a positive integer, and $M$ be a number in $\mathbb{Z}_{pq}$ with $\gcd(M, pq) = 1$. Then $M^{1+k(p-1)(q-1)} \equiv M \mod{pq}$.\\

    \textbf{Theorem 5.4:} Let $p$ and $q$ be distinct primes and $E$ be a number in $\mathbb{Z}_{(p-1)(q-1)}$ such that $E$ is relatively prime to $(p-1)(q-1)$. Then $E$ has a multiplicative inverse in $\mathbb{Z}_{(p-1)(q-1)}$. That is, there exists some $D$ in $\mathbb{Z}_{(p-1)(q-1)}$ such that $D \equiv E^{-1} \mod{(p-1)(q-1)}$.
\end{flushleft}
\begin{enumerate}
    \setcounter{enumi}{4}
    \item Find the primes $p$ and $q$ if $pq = 14,647$ and $\phi(pq) = 14,400$.
          $p$ and $q$ must be distinct, since $pq$ is not a square. Thus, $\phi(pq) = (p-1)(q-1) = pq - p - q + 1 = 14,400$. Thus, we can make a system of equations to solve for $p$ and $q$:\\
          $\left\{
              \begin{array}{lcr}
                  pq             & = & 14,647 \\
                  pq - p - q + 1 & = & 14,400
              \end{array}
              \right.$\\
          Thus, $p + q - 1 = 247$, so $p + q = 248$, so $ p = 248 - q$. We can plug this back into the first equation, so $(248-q)q = 14,647 \to 248q - q^{2} = 14,647 \to 0 = q^{2} - 248q + 14,647$. I put this into the quadratic equation to get $q = 97 \text{ or } 151$. $p$ and $q$ are interchangeable in this equation, so either $p = 151, q = 97$, or $p = 97, q = 151$.
    \item Prove Theorem 5.2 based on what you have already learned (perhaps in a previous section).\\
          Let $p$ and $q$ be distinct prime numbers, and let $M$ be a positive integer with $\gcd(M, pq) = 1$. According to Euler's Theorem, if $m$ is a positive integer and $a$ is a positive integer with $\gcd(a, m) = 1$, then $a^{\phi(m)} \equiv 1 \mod{m}$. By setting $a = M$ and $m = pq$, we get $M^{\phi(pq)} \equiv 1 \mod{pq}$. In the previous section, we also determined that where $p$ and $q$ are distinct primes, $\phi(pq) = (p-1)(q-1)$. Thus, $M^{(p-1)(q-1)} \equiv 1 \mod{pq}$.
    \item Prove Theorem 5.3 based on what you have already learned.\\
          Let $p$ and $q$ be distinct prime numbers, $k$ be a positive integer, and $M$ be a number in $\mathbb{Z}_{pq}$ with $\gcd(M, pq) = 1$. $M^{1+k(p-1)(q-1)} = M \times M^{k(p-1)(q-1)} = M \times (M^{(p-1)(q-1)})^{k} \mod{pq}$. By theorem 5.2, $M^{(p-1)(q-1)} \equiv 1 \mod{pq}$, so $M \times (M^{(p-1)(q-1)})^{k} \equiv M \times 1^{k} \equiv M \mod{pq}$. Thus, $M^{1+k(p-1)(q-1)} \equiv M \mod{pq}$.
    \item Prove Theorem 5.1 (RSA Cryptosystem Crux) based on what you have already learned.\\
          Let $p$ and $q$ be two distinct prime numbers, and let $E$ be relatively prime to $(p-1)(q-1)$. By theorem 5.4, there exists a number $D$ that is the multiplicative inverse of $E$ modulo $(p-1)(q-1)$. This means $ED \equiv 1 \mod{(p-1)(q-1)}$, or $ED = k(p-1)(q-1) + 1$ for some integer $k$. Thus, $M^{ED} = M^{k(p-1)(q-1) + 1} \equiv M \mod{pq}$ by theorem 5.3. Thus, $M^{ED} \equiv M \mod{pq}$.
\end{enumerate}\hrule \vspace{8pt}   
\textbf{The RSA Encryption Algorithm}\\ 
Suppose Alice wants to send Bob a secret message.\\
Bob picks two prime numbers, $p$ and $q$. Bob also picks an encoding number, $E$, from $Z_{(p-1)(q-1)}$ that is relatively prime to $(p-1)(q-1)$. Bob then makes $pq$ and $E$ public.\\
Alice converts her number to a message $M$ ($M$ must be less than $pq$, but this can be achieved by breaking the message into segments). To encode this message, she calculates $M^{E} \mod{pq}$. \vspace{8pt}\\ 
For these problems, suppose Bob picks $p = 3, q = 97$. Thus, $pq = 291$ and $(p-1)(q-1) = 192$. He also chooses $E = 5$.
\begin{enumerate}
    \item Suppose Alice's message is the number $M = 2$. What number does she send Bob? Describe in your own words how you found this number.\\
    First, I calculated $M^{E}$. $2^{5} = 32$. $32 \mod{291} = 32$, so she will send Bob $32$.
    \item Suppose that Alice’s secret message is the number M = 150. What number does she send Bob?\\
    Again, Alice will send Bob $M^{E} \mod{pq}$, so $150^{5} \mod 291$. By modular multiplication rules, $150^{5} \mod{291} = (150^{2})^{2}*150 = 22500^{2} * 150 = (291*77 + 93)^{2} * 150 \equiv 93^2 * 150 = 8649 * 150 \equiv 210 * 150 = 31500 \equiv 72 \mod{291}$. Alice sends $72$ to Bob.
\end{enumerate}
To decode the message, Bob must find $D$, the multiplicative inverse of $E$ in $\mathbb{Z}_{(p-1)(q-1)}$. A computer can easily compute modular inverses with the Euclidean algorithm; in this case $D=77$.
\begin{enumerate}
\setcounter{enumi}{2}
    \item Bob can now decode Alice's message.
    \begin{enumerate}
        \item Verify that 77 is indeed the multiplicative inverse of 5 in $\mathbb{Z}_{192}$. Explain in your own words how you know that you are correct.\\
        To verify, I will find $77 * 5 \mod{192}$. $77*5 = 385 = 192*2 + 1$, so $77 * 5 \equiv 1 \mod{192}$, so $5^{-1} \mod{192}$ is indeed 77.
        \item Using the RSA Cryptosystem Crux Theorem, explain how Bob can use the number $D$ to decode Alice’s encoded message $M^E$ and recover her original message $M$.\\
        By the Crux Theorem, $M^{ED} = M \mod{pq}$. The encoded message Bob receives from Alice is $M^{E}$. To find $M$ and decode the message, Bob should calculate $(M^{E})^{D} \mod{pq}$.
    \end{enumerate}
\end{enumerate}
\end{document}