\documentclass{article}
\usepackage{enumerate}
\usepackage[a4paper, margin=.5in]{geometry}
\usepackage[table]{xcolor}

\title{Submission 2.1}
% \author{Saffron Liu}
\date{}
\begin{document}
\maketitle

\begin{enumerate}
      \item In school, a curve is a function, usually from [0, 100] to [0, 100]. The following are some made-up definitions:
            \begin{itemize}
                  \item A curve f is called fair if $\forall x \forall y (x \geq y \to f(x) \geq f(y))$.
                  \item A curve f is called totally unfair if $\forall x \forall y (x > y \to f(x) < f(y))$.
                  \item A curve f is called progressive if $\forall x \forall y (x < y \to f(x) - x > f(y) - y)$.
            \end{itemize}
            \begin{enumerate}
                  \item Say (both in English and in Quantificational Logic) what it means for a curve to be unfair.\\
                        A curve is unfair if a grade x greater than or equal to a grade y before the curve ends up lower than grade y after the curve. $\exists x \exists y \neg(x \geq y \to f(x) \geq f(y))$
                  \item Say (both in English and in Quantificational Logic) what it means for a curve to be not totally unfair.\\
                        A curve is not totally unfair if a score x greater than score y does not end up lower after the curve. $\exists x \exists y \neg(x > y \to f(x) < f(y))$
                  \item Conjure an example of a curve that is unfair but not totally unfair.\\
                        An example of this would be f(x) = (x \% 2) * 100. This is not fair, because when a = 2 and b = 1 (i.e. $a \geq b$), f(a) = 0 and f(b) = 100 (i.e. $f(a) < f(b)$). However, this curve is also not totally unfair, because when a = 3 and b = 2 (i.e. $a > b$), f(a) = 100 and f(b) = 0 (i.e. $f(a) \geq f(b)$).
                  \item Say in English what it means for a curve to be progressive. Say (both in English and in Quantificational Logic) what it means for a curve to be not progressive.\\
                        A curve is progressive if the difference between the post- and pre- curve scores is higher for lower scores (for most intents and purposes, if lower scores are "bumped" more than higher scores.)\\
                        A curve is not progressive if there is a case where a lower score is bumped the same amount or less than a higher score. $\exists x \exists y (x < y \to f(x) - x \leq f(y) - y)$
                  \item Classify the curve “everyone gets 5 points” as fair or unfair and progressive or not progressive.\\
                        This curve is fair but not progressive.
            \end{enumerate}
      \item The limit of a function, f(x), at a point c is l if and only if for all $\varepsilon > 0$ there is a $\delta > 0$
            such that, for all x, if the distance between x and c is less than $\delta$, then the distance between
            f(x) and l is less than $\varepsilon$. Suppose f is a function, and the limit of f at 0 is not 1. Given
            this definition of a limit, how can you prove that the limit of f at 0 is not 1? That is, what
            evidence would you need to provide in order to prove that the limit of f at 0 is not 1?\\
            You would need to find an $\varepsilon$ such that for all $\delta$, the distance between x and 1 being less than $\delta$ does not guarantee that the distance between f(x) and 1 will be less than $\varepsilon$.
      \item A function, f, is continuous at a point c if and only if for all sequences of real numbers,
            $x_{n}$, that converge to c, the sequence $f(x_{n})$ converges to f(c). Suppose f is a function, and
            f(0) = 1. Given this definition of pointwise continuity above, what would you need to do
            to prove that f is discontinuous at 0? That is, what evidence would you need to provide in
            order to prove that f is discontinuous at 0?\\
            You would need to find a sequence of real numbers $x_{n}$ that converges to 0 but where $f(x_{n})$ does not converge to 1.
\end{enumerate}

\begin{flushleft}
      For problems 4-12:
\end{flushleft}
\begin{enumerate}[(a)]
      \item \textit{Translate the argument into quantificational logic. Be sure to delineate the extensions you give to predicates and names, and write the argument (now in quantificational logic) in standard form.}
      \item \textit{Claim whether the argument is valid or sound. In some cases, soundness will be difficult to determine, so “soundness is difficult to determine” is an appropriate answer.}
      \item \textit{Prove that the argument is valid or invalid (as appropriate) with an informal proof.}
      \item \textit{Prove that the argument is valid (if applicable) with a proof by natural deduction.}
\end{enumerate}

\begin{enumerate}
      \setcounter{enumi}{3}
      \item Everything has a cause. Therefore something is the cause of everything. (Some people think St. Thomas Aquinas advocated this.) (B 183)
            \begin{enumerate}
                  \item Cxy: x is the cause of y\\

                        $\forall x \exists y(Cyx)$\\
                        \rule{15em}{.5pt}\\
                        $\exists y \forall x(Cyx)$
                  \item This argument is invalid, and therefore unsound.
                  \item Although each thing may have a cause, they may not be the same cause. Therefore, it does not follow that all things share the same cause. For example, if A is caused by A, and B is caused by B, there is not a single thing that causes both A and B.
                  \item Because the argument is invalid, it cannot be proven with natural deduction.
            \end{enumerate}
      \item Fred hates everyone who hates Al. Al hates everyone. So Al and Fred hate each other. (B213)
            \begin{enumerate}
                  \item Hxy: x hates y\\
                        a: Al\\
                        f: Fred\\

                        $\forall x(Hax)$\\
                        $\forall y(Hya \to Hfy)$\\
                        \rule{15em}{.5pt}\\
                        $Haf \land Hfa$
                  \item This argument is valid. Soundness is difficult to determine.
                        \setcounter{enumii}{3}
                  \item \begin{enumerate}
                              \item $\forall x(Hax)$ (Premise 1)
                              \item $\forall y(Hya \to Hfy)$ (Premise 2)
                              \item $Haf$ (UI, i)
                              \item $Haa$ (UI, i)
                              \item $Haa \to Hfa$ (UI, ii)
                              \item $Hfa$ (MP, v, iv)
                              \item $Haf \land Hfa$ (Conj, iii, vi)
                        \end{enumerate}
            \end{enumerate}
      \item All insects in this house are large and hostile. Some insects in this house are impervious to pesticides. Thus, some large, hostile insects are impervious to pesticides. (B 213)
            \begin{enumerate}
                  \item Lx: x is large\\
                        Hx: x is hostile\\
                        Px: x is impervious to pesticides\\

                        $\forall x(Lx \land Hx)$\\
                        $\exists x(Px)$\\
                        \rule{15em}{.5pt}\\
                        $\exists x(Lx \land Hx \land Px)$
                  \item This argument is valid and (unfortunately) quite likely sound.
                        \setcounter{enumii}{3}
                  \item \begin{enumerate}
                              \item $\forall x(Lx \land Hx)$ (Premise 1)
                              \item $\exists x(Px)$ (Premise 2)
                              \item $Pa$ (EI, ii)
                              \item $La \land Ha$ (UI, i)
                              \item $La \land Ha \land Pa$ (Conj, iii, iv)
                              \item $\exists x(Lx \land Hx \land Px)$ (EG, v)
                        \end{enumerate}
            \end{enumerate}
      \item Some students cannot succeed at the university. All students who are bright and mature can succeed. It follows that some students are either not bright or immature. (B 213)
            \begin{enumerate}
                  \item Sx: x can succeed\\
                        Bx: x is bright\\
                        Mx: x is mature\\

                        $\exists x(\neg Sx)$\\
                        $\forall x((Bx \land Mx) \to Sx)$\\
                        \rule{15em}{.5pt}\\
                        $\exists x(\neg Bx \lor \neg Mx)$
                  \item This argument is valid. I would argue that it's unsound.
                        \setcounter{enumii}{3}
                  \item \begin{enumerate}
                              \item $\exists x(\neg Sx)$ (Premise 1)
                              \item $\forall x((Bx \land Mx) \to Sx)$ (Premise 2)
                              \item $\neg Sa$ (EI, i)
                              \item $(Ba \land Ma) \to Sa$ (UI, ii)
                              \item $\neg Sa \to \neg (Ba \land Ma)$ (ContraPos, iv)
                              \item $\neg Sa \to (\neg Ba \lor \neg Ma)$ (DeM, v)
                              \item $\neg Ba \lor \neg Ma$ (MP, iii, vi)
                              \item $\exists x(\neg Bx \lor \neg Mx)$ (EG, vii)
                        \end{enumerate}
            \end{enumerate}
      \item There are at least 3 pigs. So there are at least two pigs.
            \begin{enumerate}
                  \item Px: x is a pig\\
                        $\exists x \exists y \exists z (Px \land Py \land Pz \land (x \neq y) \land (x \neq z) \land (y \neq z))$\\
                        \rule{15em}{.5pt}\\
                        $\exists x \exists y (Px \land Py \land (x \neq y))$
                  \item This argument is valid and sound.
                        \setcounter{enumii}{3}
                  \item \begin{enumerate}
                              \item $\exists x \exists y \exists z (Px \land Py \land Pz \land (x \neq y) \land (x \neq z) \land (y \neq z))$ (Premise 1)
                              \item $\exists y \exists z (Pa \land Py \land Pz \land (a \neq y) \land (a \neq z) \land (y \neq z))$ (EI, i)
                              \item $\exists z (Pa \land Pb \land Pz \land (a \neq b) \land (a \neq z) \land (b \neq z))$ (EI, ii)
                              \item $Pa \land Pb \land Pc \land (a \neq b) \land (a \neq c) \land (b \neq c)$ (EI, iii)
                              \item $Pa \land Pb \land Pc \land (a \neq b) \land (a \neq c)$ (Simp, iv)
                              \item $Pa \land Pb \land Pc \land (a \neq b)$ (Simp, v)
                              \item $Pa \land Pb \land (a \neq b)$ (Simp, vi)
                              \item $\exists y(Pa \land Py \land (a \neq y))$ (EG, vii)
                              \item $\exists x \exists y(Px \land Py \land (x \neq y))$ (EG, viii)
                        \end{enumerate}
            \end{enumerate}
      \item Popeye and Olive Oyl like each other since Popeye likes everyone who likes Olive Oyl, and Olive Oyl likes everyone. (B 218)
            \begin{enumerate}
                  \item Lxy: x likes y\\
                        p: Popeye\\
                        o: Olive Oyl\\

                        $\forall x(Lxo \to Lpx)$\\
                        $\forall x(Lox)$\\
                        \rule{15em}{.5pt}\\
                        $Lpo \land Lop$
                  \item This argument is valid and sound.
                        \setcounter{enumii}{3}
                  \item \begin{enumerate}
                              \item $\forall x(Lxo \to Lpx)$ (Premise 1)
                              \item $\forall x(Lox)$ (Premise 2)
                              \item $Loo \to Lpo$ (UI, i)
                              \item $Loo$ (UI, ii)
                              \item $Lpo$ (MP, iii, iv)
                              \item $Lop$ (UI, ii)
                              \item $Lpo \land Lop$ (Conj, v, vi)
                        \end{enumerate}
            \end{enumerate}
      \item This argument is unsound, for its conclusion is false, and no sound argument has a false conclusion. (J 49)
            \begin{enumerate}
                  \item U: x is unsound\\
                        F: x has a false conclusion\\
                        a: this argument\\

                        $Fa$\\
                        $\forall x(Fx \to Ux)$\\
                        \rule{15em}{.5pt}\\
                        $Ua$
                  \item This argument is valid. Soundness is (very) difficult to determine.
                  \item The second premise states that any argument with a false conclusion is unsound. The first premise states that "this argument" has a false conclusion, so it must follow that "this argument" is unsound.
                  \item \begin{enumerate}
                              \item $Fa$ (Premise 1)
                              \item $\forall x(Fx \to Ux)$ (Premise 2)
                              \item $Fa \to Ua$ (UI, ii)
                              \item $Ua$ (MP, iii, i)
                        \end{enumerate}
            \end{enumerate}
      \item Everyone likes Mandy. Mandy likes nobody but Andy. Therefore Mandy and Andy are the same person. (B 238)
            \begin{enumerate}
                  \item Lxy: x likes y\\
                        m: Mandy\\
                        a: Andy\\

                        $\forall x(Lxm)$\\
                        $\forall x(x \neq a \to \neg Lmx) \land Lma$\\
                        \rule{15em}{.5pt}\\
                        $m=a$
                  \item This argument is valid and sound.
                        \setcounter{enumii}{3}
                  \item \begin{enumerate}
                              \item $\forall x(Lxm)$ (Premise 1)
                              \item $\forall x(x \neq a \to \neg Lmx) \land Lma$ (Premise 2)
                              \item $\forall x(x \neq a \to \neg Lmx)$ (Simp, ii)
                              \item $Lmm$ (UI, i)
                              \item $m \neq a \to \neg Lmm$ (UI, iii)
                              \item $Lmm \to m = a$ (ContraPost, v)
                              \item $m = a$ (MP, vi, iv)
                        \end{enumerate}
                        % Everyone (including Mandy) likes Mandy; Mandy likes nobody except Andy. Mandy must like Mandy, so Andy must be Mandy.
            \end{enumerate}
      \item Everyone is afraid of Mr. Hyde. Mr. Hyde is afraid only of Dr. Jekyll. Therefore, Dr. Jekyll is Mr. Hyde. (B 234)
            \begin{enumerate}
                  \item Axy: x is afraid of y\\
                        h: Mr. Hyde\\
                        j: Dr. Jekyll\\

                        $\forall x(Axh)$\\
                        $\forall x (x \neq j \to \neg Ahx)$\\
                        \rule{15em}{.5pt}\\
                        $h=j$
                  \item This argument is valid and sound.
                        \setcounter{enumii}{3}
                  \item \begin{enumerate}
                              \item $\forall x(Axh)$ (Premise 1)
                              \item $\forall x (x \neq j \to \neg Ahx)$ (Premise 2)
                              \item $Ahh$ (UI, i)
                              \item $h \neq j \to \neg Ahh$ (UI, ii)
                              \item $Ahh \to h = j$ (ContraPos, iv)
                              \item $h = j$ (MP, v, iii)
                        \end{enumerate}
            \end{enumerate}
\end{enumerate}

\begin{flushleft}
      For problems 13-15:
\end{flushleft}
\begin{enumerate}[(a)]
      \item \textit{Claim whether the argument is valid or invalid.}
      \item \textit{Prove that the argument is valid or invalid (as appropriate) with an informal proof.}
\end{enumerate}

\begin{enumerate}
      \setcounter{enumi}{12}
      \item \begin{tabular}{c}
                  $\forall x(Fx \to Gx)$ \\
                  $\forall x(Fx \to Hx)$ \\
                  \hline
                  $\forall x(Gx \to Hx)$
            \end{tabular}
            (B 204)

            \begin{enumerate}
                  \item The argument is invalid.
                  \item Let y be an instance of x such that Fy is false, Gy is true, and Hy is false. In this case, both $Fy \to Gy$ and $Fy \to Hx$ are true, but $Gy \to Hy$ is not. This means that the conclusion is not true for all x.
            \end{enumerate}
      \item \begin{tabular}{c}
                  $\forall x(Fx \to Gx)$ \\
                  \hline
                  $\forall x(\neg Gx \to \neg Fx)$
            \end{tabular}
            (B 204)
            \begin{enumerate}
                  \item This argument is valid.
                  \item This is a contrapositive.
            \end{enumerate}
      \item \begin{tabular}{c}
                  $\neg \exists x(Fx \land Gx)$ \\
                  $\forall x(Gx \to Hx)$        \\
                  \hline
                  $\forall x(Fx \to \neg Hx)$
            \end{tabular}
            (B 204)
            \begin{enumerate}
                  \item This argument is invalid.
                  \item Let y be an instance of x such that Fy is true and Hy is true. Since Fy is true and there is no x such that both Fx and Gx are true, Gy must be false. However, this does not imply anything abuot Hy. Since this case does not pose any contradictions but renders the conclusion false, the argument is invalid.
            \end{enumerate}

\end{enumerate}

\end{document}